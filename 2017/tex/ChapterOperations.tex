\chapter{Chapter Operations}

  The Beta Nu Chapter has an extremely well organized and documented program for chapter operations. All of our executive officers, elected or appointed positions, and committee description resources are posted on our shared Google Drive for public chapter use.
  
  \section*{Elections and Appointments}
    Beta Nu's elections take place over the course of three weeks. During the first week, nominations are accepted for all elected positions. During the second week, Executive Council elections are held, while non-executive positions are elected in the third week. Before every position is elected, the floor is opened for nominations one final time. During the election, every candidate has a set speaking and question-answer time. Our bylaws stipulate a minimum time, and by custom the candidates are limited to:
    
    \begin{itemize}
    	\item President 
	  7 minutes to speak, 5 minutes to answer questions. All unused speech time
	  can be used for questions.
	  
	\item Other EC Positions
	  5 minutes to speak, 4 minutes to answer questions. All unused speech time can
	  be used for questions.
	  
	\item Non EC Positions
	  3 minutes to speak, 2 minutes to answer questions. All unused speech time can
	  be used for questions.
    \end{itemize}
    
    After each candidate speaks, the candidates withdraw while the chapter has discussion. Once again, the bylaws give minimum times for discussion, and by custom discussions are limited to 15 minutes for President, 10 minutes for other EC positions, and 5 minutes for non-EC positions. Votes are done via a secret ballot collected by the Second Guard and the Chaplain. Any brother interested in holding an appointed position is asked to submit a letter of intent to the President detailing his plans for the position and any qualifications. The President then selects who he wants to fill the position - in conjunction with the Executive Council - and the chapter approves it during the final chapter of the semester. Installation of officers is held during the final chapter of the semester. 
    
    That is not to say the chapter room is the only place for election discussion: many conversations take place outside chapter, with candidates providing ideas and platforms, answering concerns and questions. It is generally expected of officer candidates to have met with those who have previously held the office, and have a good understanding of what the position demands.
    
    \section*{Executive Council}
      The Executive Council is composed of the President, Vice-President, Vice President of Health and Safety, Secretary, Treasurer, Marshal, Recruitment Chair, Scholarship Chair, and EC Member at Large. The Member at Large is a non-voting member of EC who has previously held an EC position; he is the only appointed member of the EC. The Council meets once a week to discuss matters of chapter operations including scholarship, finance, ritual activities, goals, and standards. EC has broad powers to take matters into their own hands, but everything must be approved by the chapter at the next regularly scheduled chapter meeting.
      
      \subsection*{President}
	The President's main role is to function as the external face of the chapter. The President is responsible for all lines of communications with the Greek Life Office, IFC, and other chapters. Within the chapter the President is responsible for big ideas and plans pushing the chapter as a whole using the Executive Council to accomplish such goals. Lastly, in case of emergencies he is the first in line to deal with issues.
	
      \subsection*{Vice President}
	The Vice President's main roles are internal. He heads both our Executive Council and Committee Chair Committee meetings every week. The primary purpose of the Vice President is to relay information about chapter progress to the President and to respective chairs when needed.
	
      \subsection*{Vice President of Health and Safety}
	The Vice President of Health and Safety has two major roles, one functioning as the internal monitor of each brothers health, both physical and mental, and two serving as the public coordinator for Sacred Purpose events for the greater community.
	
      \subsection*{Secretary}
	The Secretary's primary duties are to record the actions of the chapter, facilitate communication to the brothers, and handle all communication to the Grand Chapter and CWRU's Greek Life Office. During chapter, he records the minutes and attendance. Additionally, he receives and organizes officer reports, excuses, and other information/communication requests. He performs a broad array of organizational maintenance, including the mail closet, Google Drive, and the ``God Calendar''- so named because if an event isn't on the God Calendar, it doesn't exist. \\

	One of the programs our Secretary has continued is TWIOX- ``This Week in Theta Chi''. This is a newsletter that is sent out to the email list that collects and details different chapter happenings and brotherhood events during the next week. The newsletter often contains humorous additives.
	
      \subsection*{Treasurer}
	The role of our Treasurer is very traditional. He manages the budget, communicates with the chapter Housing Corp., and any entity the chapter deals with on a financial basis. He, in tandem with the chapter, sets the budget, along with prices for dues and rent. He oversees payment from each brother, utilizes  OmegaFi, and he handles any payment deferments that brothers request. He ensures our taxes are paid accurately and in a timely manner, and he keeps an emergency fund in the case of unforeseen events. 
      
      \subsection*{Marshal}
	The Marshal's job is to introduce each new candidate class to the fraternity, teach them our ways and history, and make them into men who are capable brothers and leaders. Each Marshal has substantial preference in how he chooses to run the position. Every semester, the Marshal writes up a plan which the chapter then reviews and votes on. 

      \subsection*{Recruitment Chair}
	The Recruitment Chairman's main role in the chapter is to plan and execute Rush as well as organize 365 recruitment events. With the ideas of deferred recruitment possibly coming to CWRU he has taken up to a more active maintenance of a names list and inviting them throughout the semester to alleviate the reliance on the rush period itself.
	
      \subsection*{Scholarship Chair}
	The Scholarship Chair works to aid the academic success of brothers - particularly those who are struggling - as well as celebrate those who are doing well.

	He enforced the study room as a dedicated academic space rather than simply another room for people to socialize. \\

	Each semester, the Scholarship Chair holds a scholarship dinner. During the fall, we hosted the dinner where we invited professors, and celebrated the brothers who have performed exceptionally. In the spring, the dinner is traditionally held at the house, which is coming up soon as of the writing of this award application. \\

	The Scholarship Chair puts brothers below the 3.0 standard on scholarship contracts, giving them individualized requirements. These requirements can include mandatory study hours, weekly reports on grades/progress, and a maintained calendar of assignments. \\
	
    \section*{Other Officers}
	The other elected officers are the Historian, Chaplain, First Guard, Second Guard, and the Standards Board Justices. Our appointed positions include the Social Chair, Alumni Relations Chair, Member Development Chair, House Manager, Assistant House Manager, Detail Manager, Philanthropy and Service Chairmen, Public Relations Chairman, Risk Manager, IFC Representative, Librarian, $\Theta$X Roast Chair, Greek Week Chair, Food Steward, and the Athletics Chair. Appointed positions may be created at will by the President to accomplish certain tasks.
	
    \section*{Local Bylaws}
	Beta Nu's local by-laws are stored on the publicly shared Google Drive our chapter uses for data management. We have held several by-law revision roundtables to facilitate the relevancy of these documents.
	
    \section*{Goals and Retreats}	
	Chapter goals are set by the chapter during the semesterly retreat. During the Retreat, topics ranging from recruitment advice to chapter finance management are discussed. \\
	
	During the spring Executive Board retreat goals were set for the Executive Council to work to acomplish in the coming year. 
	\begin{itemize}
		\item Leadership: More Effective Leaders
		  \begin{itemize}
		   \item Generate documentation templates
		   \item Develop leadership skills within committee chairs
		   \item Utilize Chapter Advisory Board more effectively
		   \item Better Leadership mentorship. For example: Candidates shadowing ec members, and avoiding having brothers cover multiple positions
		   \item Have EC delegate things to other brothers
		   \item Utilize Standards Board
		   \item Have generic minutes for closed EC topics to promote transparency
		   \item Revisit goals regularly throughout semester
		   \item Improve communication between officers
		  \end{itemize}

		\item Citizenship: Meaningful Service
		  \begin{itemize}
		   \item Improve median service hours
		   \item Continue to have good attendance at other Greek organizations' philanthropies
		   \item Investigate joint philanthropies
		   \item Decide Mental Health Awareness Week's objective (awareness, raising money, assisting individuals).
		   \item Continue the service leadership plan
		   \item Get involved in GLO restructuring plan
		   \item Have a USO philanthropy event
		  \end{itemize}

		\item Ritual: Recognize Ritual vs. ritual
		  \begin{itemize}
		   \item Pring new public rituals
		   \item Have closed roundtables to discuss ritual
		   \item Increase communication with alumni
		   \item Membership Development execution of Resolute Man Program
		   \item Involve ourselves in chapter traditions (ex. caroling, Greek Week, Homecoming Parade)
		   \item Establish relationships with new sororities on campus
		   \item Encourage culture of care with brothers
		  \end{itemize}
		  
		\item Scholarship: Holistic Scholarship \& Culture of Care
		  \begin{itemize}
		   \item Move programing aspect to Membership Development
		   \item Better records for holistic events
		   \item Organizing scholarship discussion with candidates during bid discussion
		   \item Holistic scholarship within Rush/recruitment
		   \item More professors at scholarship dinner
		   \item Follow the new changes to the bylaws
		  \end{itemize}

	\end{itemize} 
    \section*{Committee}
    
	Beta Nu has three different committee types: Operational, Standing, and Ad-Hoc committees. \\
	
	Standing committees meet every other week and consist of the Recruitment Committee, PR Committee, Service Committee, and the Social Committee. Operational committees meet monthly and consist of the Alumni Relations Committee, Membership Development Committee, and Scholarship Committee. Ad-hoc committees meet as necessary, and include such things as the budget setting committee, and bylaws revision committies. All committee chairs meet weekly for the Committee Chair Committee, which is headed by the Vice-President, in order to discuss the week’s progress and to collaborate with each other.
	
	\begin{enumerate}
	  \item The Recruitment Committee is chaired by the Recruitment Chair, and it assists him in planning recruitment for the next semester. This includes 365 recruitment events, the recruitment calendar, PR plans, and planning recruitment workshops.
		
	  \item The Public Relations (PR) Committee is chaired by the PR Chair and assists in planning PR campaigns for our events. They are also in charge of managing our brand on campus. 
		
	  \item The Philanthropy and Service Committee helps the chair organize and plan service events for the semester.
		
	  \item The Social Committee helps the chair organize social events. They work closely with the PR committee to advertise events, the Recruitment Chair to plan 365 events, and the Service Committee to plan service mixers. 
		
	  \item The Alumni Relations Committee is headed by the Alumni Relations Chair and is responsible for alumni outreach. This includes organizing the annual Christmas party, Founder’s Day celebrations, and the newsletter. It is largely through their efforts that the chapter has such a strong showing in the Roster Book Rally.
		
	  \item The Membership Development Committee, headed by the Membership Development Chair, is in charge of planning membership development mini-sessions in chapter and hosting a variety of programs to help brothers become better men.
		
	  \item The Scholarship Committee is focused on chapter grades. They assist the Scholarship Chair in providing help for struggling members. They also help plan the scholarship recognition dinner, and invite campus resources to speak with the chapter.
	  
	  \item The Health and Safety Committee is focused on planning Sacred Purpose events around campus and most importantly Mental Health Awareness Week in the fall.
	\end{enumerate}
    
    \section*{Standards Board}
      The chapter maintains a fully functional Standards Board that consists of an Arbiter, Scribe, Parliamentarian, and seven Justices. Justices cannot be members of the Executive Council. \\

      The role of Standards Board is to ensure compliance with the International Bylaws, Local Bylaws, Ritual, and campus rules. Any brother may bring any other brother to Standards, at which point the Board mediates the dispute and issues sanctions with an eye toward helping to resolve strife rather than perpetuating it. \\

      Far more commonly, the Standards Board is used to recognize brothers. They meet at least monthly to issue awards such as Brother of the Month, Officer of the Month, and Alumnus of the Month. These awards recognize the brothers who have made outstanding contributions to the chapter and exemplify the ideals of Theta Chi. Others awards may be given at the discretion of the board. A copy of our local bylaws is included within the appendices. \\
      
    \section*{Documentation and Transitions}
      When transitioning, every incoming officer is required to meet with their predecessor to discuss the specific details of the position, set goals, and get any tips. Both officers then meet with the Vice-President where they discuss their transitions. Every officer also has a notebook that is passed down that details the day to day minutia of the officers. Officers also keep any documents relevant to their position in the Google Drive. Every relevant document going back as far as 2001 is kept on this Google Drive for future usage.
      
    \section*{Internal Communications}
      The chapter maintains an email list that all brothers are a part of. Reminders and announcements are sent to this list, and it is integral to our chapter operations. We also keep an online calendar lovingly called ''The God Calendar.`` All brothers can create or edit events. Common events include chapter meetings, campus events, and so on. Additionally, the Secretary continues to send out TWIOX- \textit{This Week in Theta Chi}. We believe a major reason for the improvement in attendance of a variety of events is due to TWIOX.
	  
	  The chapter also maintains a website known as the Dashboard which stores a variety of chapter information such as contact information for all brothers and is also used to handle excuses. Many officer positions also have their own page on the Dashboard where they can preform chapter functions relevant to their position.
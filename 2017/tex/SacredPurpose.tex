\chapter{Sacred Purpose}

  Sacred purpose embodies the care and heart we show for one another through all of our actions and deeds. The new position of Vice President of Health and Safety has revolutionized our chapter's identity and our programming. We couldn't be more thankful for this new initiative that IHQ has started. \\

  In specific, sacred purpose has given us an identity that extends beyond an organization of collegiate men. In our previous and current semesters, we have accomplished all of the following:
  
  \begin{enumerate}
    \item Mental Health Awareness Week (MHA Week)
  	
  	The week long event rallied the campus community to challenge the stigmas associated with talking about mental health. By the end of the event, we had received overwhelmingly positive feedback from professors, students, and the broader CWRU community.  A picture is worth a thousand words, and we hope you have a chance to peruse our appendix. Here is a summary of some of the accomplishments of the week:
  	
  	\begin{enumerate}
	      \item We raised over \$2,000 for the National Alliance on Mental Illness (NAMI), similiar to last year.
	      \item Hundreds of students shared their struggles with us in safe environments.  		
	      \item Our community learned about various campus resources to combat depression and other mental illnesses through resources like counseling services.	      
	      \item The campus chalked the Spitball Statue, located in the heart of campus, with motivational messages.
	      \item We passed out over 1000 green ribbons that students proudly wore.
	      \item The community heard from speakers from NAMI of Greater Cleveland and the University Hospitals Mood Disorders Program
	      \item Students were able to discover and connect with the variety of campus advocates from all the student organizations through our mental health fair
	      \item Ran a ``Smashing the Stigma'' event where we had people pay to smash pumpkins and donate the money to NAMI, given it was during the Halloween season.
  	\end{enumerate}
  	
  	Since we held MHA Week our first MHA Week, new organizations have formed to address mental health on campus. This includes NAMI at CWRU, which has taken off, raising at least another \$10,000 on its own in its first year. There has been a university sponsored Mental Health Alliance, to bring mental health advocates together to coordinate efforts, as well as a student driven Mental Health Policy Reform Committee. Our action has brought on change, with others now trying to address the issues. One of the big changes moving forward that has been discusse d in the chapter is the idea of moving MHA more towards an Awareness focus and move our philnathropic efforts towards USO. We have been looking into partnering with other chapters in the community for this to combat the overprogramming problem that exists on our campus. 
  	
    \item Seminar Series
    
	  Under the guidance of Sacred Purpose, our VPHS has planned several presentations and discussions that will improve the knowledge of health and safety within the chapter. This semester, the following topics have been discussed:	
	  \begin{enumerate}
	    \item Healthy Eating: This semester, our chef discussed the means of how to eat healthy. He then provided a demonstration by cooking a healthy and simple dish
	    \item Mental Health: Delegates from NAMI at CWRU will be bringing a discussion-based seminar regarding recognizing signs of mental health issues, as well as mental health in general. This is scheduled for April 14th.
	    \item Alcohol Safety: Delegates from the Students Meeting About Risk and Responsibility Training (SMARRT) will provide a discussion-based seminar on alcohol safety. This is planned for late-April.
	    \item Men's Sexual Health: Although the date is yet to be determined, SMARRT leaders will also be bringing a discussion on men's sexual health.
	    \item ALICE (Alert, Lockdown, Inform, Counter, Evacuate) Training: Officers from the CWRU PD came to our house and gave us a training session on how to prepare for an active shooter scenario
	\end{enumerate}
	\item Suggestion Box
		\begin{enumerate}
			\item The submission box is a publicly avaiable private box that is only checked by the VPHS. It serves as a means to reach out, warn, and suggest things.
			\item The box is in the mail closet for easy access and has helped in anonymous informing of brothers' mental health.
		\end{enumerate}
 \end{enumerate}
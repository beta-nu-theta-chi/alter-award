\chapter{New Member and Member Education}

  \section*{Candidate Education}
    Our Candidate Education program is lead and executed by the Marshal of the chapter. Upon the beginning of the semester, the Marshal submits a ``Marshal Plan'' to be carried out for the education of new members. Events are planned for the candidates so that they may gain familiarity with the chapter and its members. Specifically, a service and social event are planned in addition to plaque organization, which is led and completed by the candidate class. The Fall 2015 candidate class organized a service event to the East Cleveland Cemetary for the annual Graveyard Charge cleanup. The Spring Class held a Super Smash Brothers philanthropy tournament. \\
  
    Candidate education meetings occur throughout the candidate process, and a plethora of topics are discussed and shared. One meeting is devoted to chapter history, and the organization of the fraternity is discussed. Last fall, our chapter invited alumnus members to give presentations on how Beta Nu of Theta Chi's fraternal identity has changed over the years. Another meeting invites brothers to share with candidates their positions in the chapter and what they do; afterwards, the chapter typically participates in more free spirited activities, such as S'Mores and ``Meet a Brother,'' in which candidates can spend relaxing time with brothers. \\

    Still other candidate education meetings include more somber events, such as Death of a Fraternity and the Flag Drop. Flag Drop invites brothers to share weaknesses they have seen in the chapter, and it challenges members to improve themselves lest the chapter fail. Meanwhile, Death of a Fraternity is an event led by a storyteller who shares with candidates how Sigma Beta of Michigan State University failed. Death of a Fraternity and Flag Drop is attached in the appendices.
  
  \section*{Big Brother Program}
  
    Toward the middle of the candidate process, candidates compile a ranked list of brothers they feel who would be excellent mentors in the chapter. The Marshal and past Marshals then convene and determine an appropriate Big Brother for the new member. The charge of the Big Brother is to facilitate the smooth transition of the Little Brother into the chapter. ``Bigs'' support their ``littles'' with everything from homework help to life advice; they are an essential, yet often underappreciated, role in our chapter’s survival. \\
    
    Once candidates are assigned their Big Brother, a ceremony is held to commemorate the event. Littles are then offered the opportunity to live in the house for a week, while the Big moves out of his room to reside somewhere else in the house. The goal is for Littles to gain an understanding of house life and further immerse themselves in our chapter’s members. Without a doubt, the Big Brother Move Out is one of the most anticipated events. 
   
  \section*{Member Education}
      Once candidates become members, our Ritual charges us to ensure that members are continually advanced in academic and personal education. Over the past two semesters, our chapter has endeavored to expand the usefulness and frequency of member education events through the work of the Membership Development Chair (MDC). \\
      
      During the Fall 2015 officer elections, membership development was a big theme, which included as part of the now-President's platform a ``Resolute Man Membership Development Program'', which coincidentally ended up being the same name as the program IHQ is currently proposing. The VPHS is now working on integrating the Resolute Man Program into our chapter activities.\\
      
      Some of the membership development efforts in our chapter include field trips, weekly mini-education sessions during chapter, and the creation of a chapter history museum. The mini-education sessions are 5 minute presentations where brothers can volunteer to share a life skill or piece of education in front of the chapter. \\
      
      Staple member education events continue to thrive. Our chapter holds continual ``Chaplain Chowdowns'', where the Chaplain shares campus issues with the chapter over food. Some roundtables have occurred over scholarship by laws and their interpretations. \\
      
      A major development in member education is the creation of a house museum. The museum – which is being directed under the guidance of the chapter Historian – will include two display cases consisting of old memorabilia from when our chapter was Alpha Nu of Beta Kappa, national awards, old financial ledger books from when our chapter was Sigma Tau Delta (a local fraternity at the old Case School of Applied Science), and a plethora of pictures. Specifically, pictures from 1909 – when our Sigma Tau Delta chapter was formed – to the present will line the staircase in the form of a timeline, with the base of the stairs being the earliest pictures of our brotherhood, and the top being the most recent. The museum will serve as reminder for brothers of how we evolved and how our values have lived through the years. 
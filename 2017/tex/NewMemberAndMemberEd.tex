\chapter{New Member and Member Education}

  \section*{Candidate Education}
    Our Candidate Education program is lead and executed by the Marshal of the chapter. Upon the beginning of the semester, the Marshal submits a ``Marshal Plan'' to be carried out for the education of new members. Events are planned for the candidates so that they may gain familiarity with the chapter and its members. Specifically, a service and social event are planned in addition to plaque organization, which is led and completed by the candidate class. The fall 2016 candidate class organized a social event to allow brothers and candidates to get to know each other better, while the spring class planned a philanthropy event with Community Greenhouse Project, a nonprofit urban farm that supplies locally grown food at low cost to Cleveland residents.\\
  
    Candidate education meetings occur throughout the candidate process, and numerous topics are discussed and shared. One meeting is devoted to chapter history, and the organization of the fraternity is discussed. This Spring, our chapter invited alumnus members to give presentations on how Beta Nu of Theta Chi’s identity has changed over the years, and what public ritual means to them. Another meeting invited brothers to share with candidates their positions in the chapter and what they do; afterwards, the chapter typically participates in more free spirited activities, such as making S’Mores and doing ``Meet a Brothers'', in which candidates can spend time getting to know brothers they otherwise might not meet. Still other candidate education meetings include more somber events, such as Death of a Fraternity and the Flag Drop. Flag Drop invites brothers to share weaknesses they have seen in the chapter, and it challenges members to improve themselves lest the chapter fail. Meanwhile, Death of a Fraternity is an event led by a storyteller who shares with candidates how Sigma Beta of Michigan State University failed, so that they might better understand the warning signs should they ever appear.
  
  \section*{Big Brother Program}
  
    Toward the middle of the candidate process, candidates compile a ranked list of brothers they feel who would be excellent mentors in the chapter. The Marshal and past Marshals then convene and determine an appropriate Big Brother for the new members. The charge of the Big Brother is to facilitate the smooth transition of the Little Brother into the chapter. “Bigs” support their ``littles'' with everything from homework help to life advice; they play an essential, yet often underappreciated, role in our chapter’s survival. Once candidates are assigned their Big Brother, a ceremony is held to commemorate the event. Littles are then offered the opportunity to live in the house for a week before initiation, while the Big moves out of his room to reside somewhere else in the house. The goal is for Littles to gain an understanding of day-to-day chapter life and further immerse themselves in the brotherhood. Without a doubt, Brotherhood Week is one of the most anticipated events, and truly cements the sense of shared community between brothers and soon-to-be brothers.
   
  \section*{Member Education}
      Once candidates become members, our Ritual charges us to ensure that members are continually advanced in academic and personal education. Our chapter has endeavored to expand the usefulness and frequency of member education events through the work of the Membership Development Chair (MDC). These include both Resolute Man goals, and separate events designed to foster chapter unity and growth. The VPHS is now working on integrating the Resolute Man Program into our chapter activities, with the help of the Marshal and MDC. Some of the membership development efforts in our chapter have included field trips, mini-education sessions taught by brothers, and the creation of a chapter history museum. Staple member education events continue to thrive. Our chapter holds regular brotherhood roundtables, where the Chaplain hears concerns and facilitates conversation about chapter and university issues. Some recent roundtables have concerned updated scholarship bylaws, attendance policies and Resolute Man compliance. A major development in member education is the creation of a house museum. The museum, which was assembled under the guidance of our chapter Historian includes two display cases consisting of old memorabilia from when our chapter was Alpha Nu of Beta Kappa, national awards, old financial ledger books from when our chapter was Sigma Tau Delta (a local fraternity at the old Case School of Applied Science), and a plethora of pictures and mementos from our early history. Specifically, pictures from 1909 when our Sigma Tau Delta chapter was formed to the present now line the staircase in the form of a timeline, with the base of the stairs being the earliest pictures of our brotherhood, and the top being the most recent. Pictures continue to be added to our timeline, and the museum now stands as a proud record of our history both as a local chapter and nationally.
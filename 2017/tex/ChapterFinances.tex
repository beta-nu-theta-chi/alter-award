\chapter{Chapter Finances}

  Beta Nu's finances have remained strong. Omega Fi's billing and collections services have allowed exact record keeping, account management, and easy member payment. Coupled with our own locally maintained records of income and expenses, Beta Nu is able to plan out its budget each year and predict the necessary adjustments.

  \section*{Financial Statistics}
    This has allowed the chapter's finances to continue to prosper. The chapter has had a 94.2\% collection rate (Fall 2016) including brothers on payment plans who complete their payments over time. Each year, our collection rate from active brothers ranges anywhere from 95-100\% (since Spring 2011- until Fall 2016). With our housing lease costing us around \$84,000 each semester, we must be vigilant in collecting money and paying off all bills. \\

    At the present time, the chapter checking account contains \$13,200.21 with \$3672.23 kept in an emergency fund (4\% of our total net income needed). In late 2013-2014 our financial advisor, John Boyer, suggested that the amount of money in the account at the time (90k+) was too large. Accordingly, he encouraged us to make capital purchases to improve the house. We have been looking to refinish the kitchen.
    
  \section*{Assets}
    Our housing corporation has dollar amounts in the upper six figures, and they will cover any shortcomings and needs that the chapter isn't able to fund in the foreseeable future. Right now, our chapter is an excellent fiscal health, but if there is ever a rainy day, we have the alumni support and commitment to protect us. An additional asset that we have is our chapter specific scholarship fund, managed by IHQ, with \$14,000 as of last September. As mentioned in the scholarship section of this application, we are awarding scholarships to brothers for the first time in at least a few years. We now have an excellent grasp of all our assets.
    
  \section*{Budgeting}
    Budgeting and Billing is planned for every academic year. The budget is comprehensive, covering all anticipated expenses for the next fiscal year. Every Spring, the Treasurer chairs the Financial Committee, and together they formulate a budget. When creating this budget, the committee examines all historical information (we keep the previous seven fiscal year’s worth of data), which includes prior budgets and bills, and forecast any changes that will affect the budget for the upcoming year. Once completed, the Treasurer presents the budget to the Chapter Financial Advisor, Alumni Corp. Treasurer, and the chapter as a whole for approval. The budget from the 2015-2016 school year is attached, as is the approved budget for the 2016-2017 school year. \\

    The Treasurer uses the budget and provides the chapter with a schedule of charges for the upcoming semester that breaks down the rent, dues, and meal plan, as well as the billing due dates for the semester. There are three bills issued each semester, each four weeks apart. The first bill covers one half the total cost, the second one quarter, and the third one quarter. Any miscellaneous debits or credits are added throughout the semester after the credit or debit is incurred.
    
  \section*{Fiscal Responsibility}
  
    The Beta Nu chapter has continued its culture of fiscal responsibility. Every member is still expected to pay their bills on time, and many do without reminders from the Treasurer. If a member is unable to pay, he must submit a detailed deferment with a payment plan to the Executive Council for approval. Any member who does not submit a deferment, or who submits an invalid deferment, is placed on financial probation. This results in a suspension of all social and voting privileges. After a week, financial probation leads to a suspension trial. Members who are deferring more than \$1,000; deferring for periods of time when school is not in session (e.g. summer); or are graduating must fill out and sign a promissory note to ensure that the debt is properly recorded in a legally binding manner. \\

    The Treasurer is responsible to maintain the strong financial base for the chapter. The Assistant Treasurer acts as an aid in these matters and as an extra set of eyes over the finances. The Treasury pays all debts and bills in a timely manner to maintain the chapter's good name and credit. There are a wide variety of costs, including insurance, the annual membership fee, semester IFC Dues, rent, meal plans, utilities, and all other bills received. Where possible, automatic online payments have been scheduled and maintained to ensure payment. Before each payment is made, the Treasurer examines the invoice, ensuring that it is correct and that the chapter checking account will not overdraft. The Treasurer may cancel any payment if necessary. \\

    Officers are given a budget from which they may draw aid to fulfill their duties. Some examples include the Social Chair using his budget to host mixers with other Greek organizations, the Service and Philanthropy Chair giving reimbursing attendance and philanthropy events, and the Alumni Chair to create a newsletter. A meticulously itemized record of all expenses and receipts are kept locally and online. In addition, we follow the reimbursement policy to ensure we follow all financial guidelines for maintaining our 501C(7) status. \\

    The chapter has a culture of financial responsibility that starts from the candidate process. The Treasurer explains all of the proper procedures for paying or deferring bills and claiming receipts during the Marshal process. The penalties for not paying that include probation, suspensions, and eviction. These potential consequences are explicitly stated. It is made very clear that the payment of bills is an absolute requirement of every brother. To make sure that all brothers have a reference for interacting with the Treasury, an online ``Treasury 101`` document was created that all members have access to that describes everything the member needs to know. \\

    To foster financial responsibility among the chapter, the Treasury operates with complete transparency. At any time, a brother can see the current state for the chapter finances. The reports that a brother can see include any chapter budget and an itemized record of all purchases and receipts issued, including what was purchased, by whom, what budget it came out of, and how much it was. This does not include individual members' accounts. Any other aspect of the chapter's finances, such as collections rate, bank account balance, invoices, or anything a member can think of is available at a moment's notice. The chapter protects itself against embezzlement using the International Headquarters recommendations, including not having a chapter credit card, the avoidance of a petty cash system, and requiring two signatures on every check signed by the chapter. By custom, the President, Vice-President, or Secretary will provide the second signature. Additionally, all actions the Treasurer takes are reviewed by the President and the chapter’s financial advisor, especially when financially significant. \\

    Overall, our finances are in a great state and we have the safeguards and strategies in place to ensure we survive as a chapter even if we face problems in the future. Our alumni support is a major part of this safety net through the housing corporation, as is our auxiliary savings accounts.
    
    \section*{Officer Card}
    
    Recently we got one of the OmegaFi officer cards, which has been overall covering larger expenses and as a result decreasing turn around time with regards to reciepts.
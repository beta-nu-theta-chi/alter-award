\chapter{The Ritual}
  
  The Beta Nu chapter takes great pride in the ritual of Theta Chi, striving to practice it in our everyday lives both individually and as a group, our formal chapter programming, and our attitude on campus.

  \section*{Mental Health Awareness Week}
    Coming with the beginning of the Sacred Purpose movement, our chapter and VPHS thought about what
issues were most important to us and how we could better lend an assisting hand. We identified mental
health as a major issue that has affected many campus members, friends, and on a more personal level, our brothers. Largely ignored as a campus issue, we instituted a Mental Health Awareness Week- a weeklong set of programs to help people realize that if they were suffering, they were not alone. If they needed help, that there are campus resources out there for them. We helped people realize how common mental illness is, that they very likely have friends suffering from it, and that it should not be stigmatized. Lastly, our programming has helped raise money for the National Alliance on Mental Illness (NAMI) of Greater Cleveland. In the first year- 2014, we raised \$1000. Last year, we are proud to say that we doubled that to \$2000.  This year, we raised \$2100. Because of our efforts, there has been a major shift in campus culture, with new student organizations popping up to address the gap of mental health advocacy, and mental illness is now a prominent part of campus conversation.
    
  \section*{Awards}
    In addition to the numerous awards and endowments given to us by IHQ, our own Greek Life Office at Case Western has recognized our efforts in ritual practice. The Agnar Pytte Cup annual chapter development evaluation and interview process found Theta Chi to demonstrate “Excellence” in the Ritual pillar, with specific laudations towards Alumni Relations, Membership Development, and Incorporation of Purpose. \\
    
  \section*{Ritual Traditions}
   \subsection*{Ritual Minute}
      The Beta Nu chapter realizes that Ritual is something that is not to be practiced once a semester, as Initiation approaches, but most be discussed and well considered for us to gain the most out of it. Therefore, we hold a ritual minute during each of our chapter meetings. Similar in style to the Critic’s Report, each week the President appoints someone to give next week’s Ritual Minute. During this time, a brother presents to the chapter his interpretation and application of a portion of the ritual, providing valuable perspective on a regular basis. A sample ritual minute is attached for reference. \\
    
    \subsection*{Ritual Reviews and Brotherhood Week}
      In the weeks approaching Initiation, we begin holding Ritual Reviews; all actives are required to
attend at least one. Those with specific roles in the Initiation Ceremony will attend more. These are held by the Marshal, the Chaplain, and the First Guard. These involve readings of the ritual and discussions of its history and significance. The Chaplain and First Guard are always in attendance to ensure the protection and secrecy of the ritual materials. We take very good care of our ritual materials. Everything provided to us by IHQ is in pristine condition, and we have made improvements to other chapter ritual materials, which can be shown or described upon request to initiated brothers. In the week preceding Initiation, the candidates are invited to move into the chapter house, where they take the bed of their Big Brother. We hold a number of events based on chapter tradition. This includes Flag Drop, Death of a Fraternity, the Candidate Exam, and beginning this year, a final “Meet-a-Brother” meeting. This culminates, of course, with Initiation. \\

	The Flag Drop involves brothers holding up the fraternity flag and dropping it on account of a lack of brotherhood participation. Each brother in succession then picks up the flag on the promise to be a better brother. Death of a Fraternity features a story following the disintegration of the Sigma Beta Fraternity at MSU. The final ``Meet-a-Brother'' meeting consists of the candidates individually being led to a room where they shall meet their last brother, which they discover is them in a mirror. Initiation itself is held at a private location. All active brothers are required to attend, barring special circumstances. We are fortunate to have notable alumni and brothers of other local chapters in attendance.\\
      
    \subsection*{Ceremonies}
    In addition to Initiation, we hold a Big Brother Ceremonies, New Member Ceremonies, and a Memorial Ceremony. The Fall Big Brother Ceremony is traditionally held at our annual $\Theta$X Roast along with the Memorial Ceremony for fallen brothers. The Spring Big Brother Ceremony is held at the house. We take pride in our public ritual ceremonies, which frequently have many non-brothers in attendance. 
    
  \section*{Projects}
    From time to time, a brother becomes inspired to go about a project to bring honor to our traditions and ideals. In the past, this included investigating the history of the chapter’s houses and miscellaneous moves over time. It has included poring over chapter minutes going back to the ’60s, speaking with many brothers, to be able to construct a ``Brotherhood Tree'' that shows the relations of bigs/littles for everyone in the chapter- which is maintained still.
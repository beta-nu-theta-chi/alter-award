\chapter{Recruitment Programming}

  The Beta Nu Chapter had a reasonably successfully year of recruiting. In Fall 2015, we initiated 5 brothers under Marshal Peyton Turner, with another candidate to hopefully be initiated soon. He was unable to be initiated in the fall for personal reasons, but we hope to initiate him during our next retreat. In the Spring of 2016, under Marshal Nikhil Edward, we were able to initiate 7 new brothers. We expect that both of this year’s classes to take on leadership roles in the coming semesters. These classes bring our current chapter size to 42 members, which is slightly above average for the university.
    
  \section*{Rush}
    At Case Western Reserve, the Greek community spends the first two weeks of each semester recruiting new members. As such, Beta Nu's Recruitment Committee spends the previous semester preparing a rush calendar of events as drafted by the Recruitment Chair. The committee meets every two weeks, where they plan out and draft a rush calendar. Each event during rush is led by one captain, usually an older member of the chapter who is in charge of running the event, and a lieutenant, usually a younger member who is tasked with learning about the event and how to run it smoothly. In addition to classic favorite events, such as Havana Nights and OX Cafe, the committee reviews how events went during previous years, changing or revamping them as need be.

  \section*{365 Recruitment}    
    The Recruitment Committee also plans continual 365 recruitment events. These allow us to continue advertising Theta Chi to other PNMs, give us opportunities to keep making touches on people we liked during rush, and build relationships with people in a way that the two week rush process does not always allow. This year, our recruitment chairman has attempted to improve our ability to recruit upperclassmen by being more aggressive about hosting 365 recruitment events. This semester, held we held a pair of trivia events, as well as plans to turn our end-of-the-year Damage event into a slightly more recruitment-focused event. We hope that this will help us recruit upperclassmen in the coming recruitment season, and we intend to have four of these events in the Fall 2016 semester.
    
  \section*{Recruitment Workshops}
    Recruitment workshops are held once a semester by the Recruitment Committee, and these workshops highlight the recruitment values of OX. Highlights of the workshop include discussions on how prospective members approach Greek life in general, as well as the pressures they may face to join Greek Life or not. Beta Nu strives to create comfortable and approachable environments during recruitment. These workshops help us to better interact with prospective members. The Fall 2015 workship was given to us by Brother Neumann, while the Spring 2016 brought back a speaker we have greatly enjoyed- Marquez Brown of Delta Chi.

    
  \section*{Needs and Thoughts for Improvement}
    Due to the setup of rush at the university, as well as the number of Greek organization on campus, it would be unusual for us to be able to initiate more than seven new brothers in a semester. We do not believe that we deserve recognition in recruitment due to the relatively small number of candidates. \\
    
    In general, our chapter has been good at finding people whom we think would fit in, both during rush events and during general campus life. Once identified, these individuals are added to the names list, with whatever contact information we can muster for reference. Unfortunately, we seem to sometimes have an issue with turning people into the names list into people who show up to events and eventually accept bids. This is partly an issue with brothers not being responsible with staying in contact with their PNMs, but partly also simply due to the nature of people who would join the Beta Nu chapter— people who are often busy, and aren’t always able have recruitment events as a first priority. \\
    
    This semester, we began to address this issue by attempting to “pre-close” people whom we thought were likely to accept bids. To pre-close a PNM is to have a brother close with a PNM whom we expect to receive a bid and ask them if they had any particular questions about the chapter, as well as ask what potentially would prevent them from accepting a bid, so that we can address issues such as time balance and finances if it could be what stops someone from accepting a bid. \\
    
    Part of our attempt to improve our recruiting ability is to host more recruitments events. This semester, that has taken the form of 365 events, but for the coming semester, the recruitment chair is hoping to also organize a weekly or biweekly small event, probably a board game night or a movie night. This should allow us to invite PNMs that we think would be a good fit to the chapter to come to house on a more regular basis, so that when a brother meets someone who might be a good fit, they aren’t waiting until a 365 event or rush to start meeting people. While we believe that this could do a lot to help recruit upperclassmen and spring recruitment, we are a bit concerned about finding the manpower to run this. However, we believe that by attempting to organize this on a week-by-week basis, we could alleviate this issue by ensuring that no one had to commit to anything more than a week in advance, so that they can volunteer around long-term plans but plan short-term plans around it. \\
    
    Over the next two semesters, we hope to recruit about 20 new brothers. Due to the large number of graduating seniors this year, we intend to fill their place with a slightly higher than average number of new brothers in order to maintain our chapter size. However, we have no desire to lower our qualifications for our brothers - we will continue to look for good scholars, involved campus citizens, and outstanding gentlemen to recruit to the chapter. This is a delicate balance that we will have to find, but we believe that it will be better to prioritize quality over quantity.

\chapter{Risk Management Practices}

  At the Beta Nu chapter, we recognize that there are many personal risk factors, in addition to those that cause major chapter problems. Mental health problems are extremely common on college campuses, and even more so at CWRU. There are a number of steps that the chapter has taken to mitigate these risks and to promote good mental health among the members. First and foremost is the how the chapter has expanded the role of Chaplain. The Chaplain operates as a mediator and a counselor in conjunction with the Vice President of Health and Safety. 
  
  All brothers are expected and required to do the following: 
  
  \begin{enumerate}
  	\item Know and obey FIPG Guidelines, OX Risk Management Protocols, and CWRU Housing regulations. 
  	
  	\item How to prevent an emergency.
  	
  	\item How to respond to an emergency.
  	
  	\item How to recognize and report any hazing on campus.
  	
  	\item Minimize risk to the house, chapter, brothers, and non-members. 
  	
  	\item Follow all local, state, and federal laws.
  \end{enumerate}

  One of the most important aspects of risk management is that we voluntarily operate an alcohol-free house and host alcohol-free events. We began doing this in 1990, and it has become an integral aspect of our chapter. 100\% of the current members said that they would not have joined the current chapter if it was not voluntarily dry. More than membership though, being dry conveys an attitude. The house is a safe place to rest, relax, and study. It is not a place for parties. Having dry events also conveys an attitude that these events are for everyone to participate, not for half the chapter to have fun and the other half to watch over them. We want to ensure our living environment, where we spend the majority of our time studying/resting, is held to a high standard. \\

  In regards to emergency preparedness, the Risk Manager is in charge of making sure that the fraternity is properly prepared to respond to an emergency situation. This includes making sure fire evacuation instructions are posted and that the emergency response guide and emergency contact list are posted by every exit from the house and by every house phone. He is also in charge of coordinating preparedness training such as SMARRT meetings. This preparedness paid major dividends all of last year. For instance, when sheets caught on fire in the dryer, the members were able to quickly cut power in the basement, and fire crews came in to clean out the dryer and air out the area. \\
  
  Another major effort Beta Nu has been active in is the CWRU Greek Life Sexual Misconduct Prevention plan. None of our brothers prompted a campaign to tackle this campus-wide issue, but we are now active in learning about bystander intervention and assisting our Greek community. The attached plan was developed, in part, by our brothers. We are proud of this contribution and the continued work our members put into developing strategies to make Greek life as a whole safer for everyone involved inside and outside of it. \\

  Finally, it would behoove us to not reiterate our Mental Health Awareness Week. This embodiment of the sacred purpose has done wonders to support the morale of our campus community and of our brothers. We know that ``you are not alone'', no matter the struggles one faces in college. We talk more about this week in the sacred purpose section, but ultimately championing mental health awareness is a great way to prevent mental health problems and destroy the stigmas that surround it. \\
  
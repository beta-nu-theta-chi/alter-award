\chapter{Sacred Purpose}

  Sacred purpose embodies the care and heart we show for one another through all of our actions and deeds. The new position of Vice President of Health and Safety has revolutionized our chapter's identity and our programming. We couldn't be more thankful for this new initiative that IHQ has started. \\

  In specific, sacred purpose has given us an identity that extends beyond an organization of collegiate men. In our previous and current semesters, we have accomplished all of the following:
  
  \begin{enumerate}
  	\item Mental Health Awareness Week (MHA Week)
  	
  	The week long event rallied the campus community to challenge the stigmas associated with talking about mental health. By the end of the event, we had received overwhelmingly positive feedback from Yik Yak (the social media app - see attached pictures in the appendices), professors, and the student community.  A picture is worth a thousand words, and we hope you have a chance to peruse our appendix. Here is a summary of some of the accomplishments:
  	
  	\begin{enumerate}
	      \item We raised over \$1,000 for the National Alliance on Mental Illness (NAMI).
  		
	      \item Hundreds of students shared their struggles with us in safe environments.
  		
	      \item Our community learned about various campus resources to combat depression and other mental illnesses through resources like counseling services.
	      
	      \item We hosted a charity frisbee tournament.
	      
	      \item The campus chalked the Spitball Statue, located in the heart of campus, with motivational messages.
	      
	      \item We passed out over 1000 green ribbons that students proudly wore.
	      
	      \item The community heard stories from NAMI in our culminating charity dinner.
  	\end{enumerate}
  	
  	Since we held MHA Week, new student organizations, such as Active Minds and NAMI CWRU, have started at our university. These institutions specifically told our brothers that we inspired them to organize and continue making an impact at CWRU. This is but a taste of what our MHA Week has inspired and created at the university. Just recently, Delta Upsilon and Beta Theta Pi wanted our VPHS to do a roundtable on mental health awareness for them. We look forward to continuing MHA Week as a new tradition of Beta Nu.
  	
  	\item SafeZone Training
	  
	  Safe Zone training taught our brothers and guests about creating a friendly and comfortable atmosphere for those in the LGBTQ communities. Specifically, we learned about gender pronouns and how to create an environment that makes everyone feel inclusive, regardless of their background. This was led by our membership development chair, and we are proud to have many brothers who are now certified as SafeZone Trained - including many elected officers.
	  
  \end{enumerate}
  
  MHA Week and Safe Zone Training are but a sample of the items we have accomplished in line with our sacred purpose. A large variety of the service and philanthropic pursuits that we participate in as a chapter have also embodied our values and our sacred purpose, and that section elaborates on initiatives that include Kids Against Hunger Cleveland. We also took the time to run a program to support the United Service Organization where people throughout our campus wrote over 50 letters for soldiers overseas and made over 30 paracord bracelets. We look forward to sending these and creating a smile for our men and women at arms.